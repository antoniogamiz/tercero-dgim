\documentclass[12pt]{article}

\usepackage[spanish]{babel}
\decimalpoint % to render points as points for decimal separators

\usepackage[utf8]{inputenc}

\usepackage[left=1cm, right=1cm, top=1cm]{geometry} 

\usepackage{amsmath}
\usepackage{amssymb}
%\usepackage{upgreek}
\usepackage{xcolor}
\usepackage{eurosym}
\usepackage{graphicx}
%%%%%%%%%%%%%%%%%%%%%%%%%%%%%%%%%%%%%%%%%%%%%%%%

\newcommand{\R}[1][]{\mathbb{R}^{#1}}
\newcommand{\N}[1][]{\mathbb{N}^{#1}}
\newcommand{\solution}[1]{\text{\fbox{$#1$}}}

\DeclareMathOperator\arctanh{arctanh}

\newcommand{\abs}[1]{\left|#1\right|}

\newcommand{\FVCP}[5]{x(t)=e^{#1}\int_{#4}^{t}e^{#2}#3ds+#5e^{#1}}
\newcommand{\FVC}{x(t)=e^{\int_{t_0}^{t}a(s)ds}\int_{t_0}^te^{-\int_{t_0}^{s}a(z)dz}b(s)ds+x_0e^{\int_{t_0}^{t}a(s)ds}, \quad x(t_0)=x_0}


\newcommand{\tick}{\textbf{\color{green}{ (\checkmark) }}}
\newcommand{\warning}{\textbf{\color{red}{ {\fontencoding{U}\fontfamily{futs}\selectfont\char 66\relax} }}}

\newcommand{\qued}{\hfill$\blacksquare$}

\newcommand{\p}[2]{\frac{\partial#1}{\partial#2}}


%%%%%%%%%%%%%%%%%%%%%%%%%%%%%%%%%%%%%%%%%%%%%%%%

\newenvironment{aclaration}    
{
\begin{center}
\begin{tabular}{|p{0.9\textwidth}|}
\hline \\ \warning
}{
\\\hline
\end{tabular} 
\end{center}
}

%%%%%%%%%%%%%%%%%%%%%%%%%%%%%%%%%%%%%%%%%%%%%%%%

\begin{document}

%%%%%%%%%%%%%%%%%%%%%%%%%%%%%%%%%%%%%%%%%%%%%%%%

\author{Antonio Gámiz Delgado}
\title{Relación III: Ecuaciones Diferenciales}
\maketitle

\begin{enumerate}

\hrule
\item Calcula, si es posible, una función potencial para las siguientes parejas de funciones. Especifíca en cada caso el dominio en el que se trabaja.
\hrule
\begin{enumerate}
\item $P(x,y)=x+y^3, \enskip Q(x,y)=\frac{x^2}{2}+y^2$
\[
\p{P}{y}=3y^2\neq x=\p{Q}{x} \Longrightarrow \text{ No existe potencial  }
\]

\item $P(x,y)=\frac{1}{2}\sin 2x-xy^2, \enskip Q(x,y)=y(1-x^2)$
\[
\p{P}{y}=-2xy \neq -2xy=\p{Q}{x} \Longrightarrow \text{Se puede calcular el potencial}
\]

\[
\p{\U}{x}=\
\]
\end{enumerate}
\end{enumerate}


\end{document}