\documentclass[12pt]{article}

\usepackage[spanish]{babel}
\decimalpoint % to render points as points for decimal separators

\usepackage[utf8]{inputenc}

\usepackage[left=2cm, right=2cm, top=1cm]{geometry} 

\usepackage{amsmath}
\usepackage{amssymb}
%\usepackage{upgreek}
\usepackage{xcolor}
\usepackage{eurosym}
\usepackage{graphicx}
\usepackage{amsthm}
%%%%%%%%%%%%%%%%%%%%%%%%%%%%%%%%%%%%%%%%%%%%%%%%

\newcommand{\R}[1][]{\mathbb{R}^{#1}}
\newcommand{\N}[1][]{\mathbb{N}^{#1}}
\newcommand{\solution}[1]{\text{\fbox{$#1$}}}

\DeclareMathOperator\arctanh{arctanh}

\newcommand{\abs}[1]{\left|#1\right|}

\newcommand{\FVCP}[5]{x(t)=e^{#1}\int_{#4}^{t}e^{#2}#3ds+#5e^{#1}}
\newcommand{\FVC}{x(t)=e^{\int_{t_0}^{t}a(s)ds}\int_{t_0}^te^{-\int_{t_0}^{s}a(z)dz}b(s)ds+x_0e^{\int_{t_0}^{t}a(s)ds}, \quad x(t_0)=x_0}


\newcommand{\tick}{\textbf{\color{green}{ (\checkmark) }}}
\newcommand{\warning}{\textbf{\color{red}{ {\fontencoding{U}\fontfamily{futs}\selectfont\char 66\relax} }}}

\newcommand{\U}{\mathcal{U}}

\newcommand{\qued}{\hfill$\blacksquare$}
\newtheorem*{theorem*}{Ejercicio}

\newtheorem*{proof*}{Solución}
\newcommand{\p}[2]{\frac{\partial#1}{\partial#2}}

%%%%%%%%%%%%%%%%%%%%%%%%%%%%%%%%%%%%%%%%%%%%%%%%

\newenvironment{aclaration}    
{
\begin{center}
\begin{tabular}{|p{0.9\textwidth}|}
\hline \\ \warning
}{
\\\hline
\end{tabular} 
\end{center}
}

%%%%%%%%%%%%%%%%%%%%%%%%%%%%%%%%%%%%%%%%%%%%%%%%

\begin{document}

%%%%%%%%%%%%%%%%%%%%%%%%%%%%%%%%%%%%%%%%%%%%%%%%

\author{Antonio Gámiz Delgado}
\title{Ejercicio Propuesto}
\maketitle

%%%%%%%%%%%%%%%%%%%%%%%%%%%%%%%%%%%%%%%%%%%%%%%%

\begin{theorem*}
Demostrar que:
\[
\{ c_1\cos t +c_2 \sin t, \enskip c_1,c_2\in\R \} = \{ A\cos (t+\theta), \enskip A\geq 0,\theta \in [0,2\pi[ \}
\]
\end{theorem*}

\begin{proof*} 
Veámos que uno está incluído en el otro y viceversa:
\begin{itemize}
\item $\supseteq$ : Usando que 
\[
\cos(a+b)=\cos a \cos b - \sin a \sin b \Longrightarrow
\]
\[
\Longrightarrow A\cos (A+\theta) = A\left( \cos t \cos \theta - \sin t \sin \theta \right) = A\cos \theta\cos t - A\sin\theta\sin t \Longrightarrow
\]
\[
\Longrightarrow c_1 = A\cos\theta \text{ y } c_2 = -A\sin\theta
\]
\item $\subseteq$: Usando lo anterior, si demostramos que podemos expresar cualquier $(c_1,c_2)$ de la forma ($A\cos t, -A\sin t$) habremos terminado.
\[
\left \{ \begin{array}{ll}
c_1 &= A\cos \theta \Longrightarrow A = \displaystyle\frac{c_1}{\cos\theta} \Longrightarrow A = \frac{c_1}{\cos\left(\arctan\left(-\displaystyle\frac{c_2}{c_1}\right)\right) } = \frac{c_1}{\sqrt{1+\left(\frac{c_2}{c_1}\right)^2}}=\frac{c_1^2}{\sqrt{c_1^2+c_2^2}}\\
c_2 &= -A\sin \theta \Longrightarrow c_2 = -c_1\tan\theta \Longrightarrow \theta = \arctan\left(-\displaystyle\frac{c_2}{c_1}\right)
\end{array}\right.
\]
Luego si tenemos $c_1\cos t + c_2\sin t$, con $A=\frac{c_1^2}{\sqrt{c_1^2+c_2^2}}$ y $\theta=\arctan\left( -\frac{c_2}{c_1} \right)$ se cumple la inclusión. \qued

\end{itemize}
\end{proof*}

\end{document}