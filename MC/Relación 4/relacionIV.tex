
% plantilla obtenida de: https://www.overleaf.com/19886281jjqffwsxshmm#/73112823/

\documentclass[a4paper, 11pt]{article}
\usepackage{comment} % enables the use of multi-line comments (\ifx \fi) 
\usepackage{lipsum} %This package just generates Lorem Ipsum filler text. 
\usepackage{fullpage} % changes the margin

\usepackage[spanish]{babel}
\usepackage[utf8]{inputenc}

\usepackage{graphicx}

\usepackage{amsmath}
\usepackage{amsfonts}
% or
\usepackage{amssymb}
\usepackage{tikz}
\usepackage{array}
\newcolumntype{C}{>{$}l<{$}} % math-mode version of "l" column type
\newcommand{\imageins}[4]{\begin{figure}[!ht]		%Take the hardwork from using images. Let this command do the work for you. Insert images by just using this command \imageins{filename}{width as a ratio of total text width of the page}{caption name}{label name for referring in articles}		
    \centering
    \includegraphics[width=#2\textwidth]{#1}
    %\caption{#3}
    %\label{#4}
    \vspace{0.2em}
\end{figure}}

%%%%%%%%%%%%%%%%%%%%%%%%%%%%%%%%%%%%%%%%%%%%%%%%%%%%%%%%%%%%%%%%%%%%%%%%%%%%%%%%%%
\usepackage{listings}
\usepackage{color}
 
\definecolor{codegreen}{rgb}{0,0.6,0}
\definecolor{codegray}{rgb}{0.5,0.5,0.5}
\definecolor{codepurple}{rgb}{0.58,0,0.82}
\definecolor{backcolour}{rgb}{0.95,0.95,0.92}
 
\lstdefinestyle{mystyle}{
    backgroundcolor=\color{backcolour},   
    commentstyle=\color{codegreen},
    keywordstyle=\color{magenta},
    numberstyle=\tiny\color{codegray},
    stringstyle=\color{codepurple},
    basicstyle=\footnotesize,
    breakatwhitespace=false,         
    breaklines=true,                 
    captionpos=b,                    
    keepspaces=true,                 
    numbers=left,                    
    numbersep=5pt,                  
    showspaces=false,                
    showstringspaces=false,
    showtabs=false,                  
    tabsize=2
}
 
\lstset{style=mystyle}

%%%%%%%%%%%%%%%%%%%%%%%%%%%%%%%%%%%%%%%%%%%%%%%%%%%%%%%%%%%%%%%%%%%%%%%%%%%%%%%%%%

\begin{document}
%Header-Make sure you update this information!!!!
\noindent
\large\textbf{Relación IV} \hfill \textbf{Antonio Gámiz Delgado} \\
\normalsize Modelos de Computación \hfill 21/11/2018
%\normalsize ECE 100-003 \hfill Teammates: Student1, Student2 \\
%Prof. Oruklu \hfill Lab Date: XX/XX/XX \\
%TA: Adam Sumner \hfill Due Date: XX/XX/XX

\section*{Ejercicio 14}
Dar gramáticas independientes del contexto que genren los siguientes lenguajes sobre el alfabeto $A=\{0,1\}$:
\begin{enumerate}
\item $L_1$: conjunto de palabras tal que si la palabra empieza por 0, entonces tiene el mismo número de 0s que de 1s:
\[S\longrightarrow 0A_0|1D\]
\[ D \longrightarrow BD|\varepsilon\]
\[ B\longrightarrow 0|1\]
\[ A_0\longrightarrow 0A_x|1A_0A_0\]
\[ A_1\longrightarrow 1A_x|0A_1A_1\]
\[ A_x\longrightarrow 1A_0|0A_1|\varepsilon\]
\item $L_2$: conjunto de palabras tal que si la palabra termina por 1, entonces tiene un número de 1s mayor o igual que el número de 0s:
\[ S\longrightarrow D0|C1\]
\[ D\longrightarrow BD|\varepsilon\]
\[ B\longrightarrow 0|1\]
\[ C\longrightarrow \varepsilon|0|C1C\]
\item $L_1\cap L_2$:
\[ S\longrightarrow 0A_0|1E \]
\[ E\longrightarrow CC1|D0\]
\[ A_0\longrightarrow A_0A_0\]
\[ A_1\longrightarrow 1A_x|0A_1A_1\]
\[ A_x\longrightarrow 1A_0|0A_1|\varepsilon\]
\end{enumerate}
\section*{Ejercicio 16}
Una gramática independiente del contexto generalizada es una gramática en el que las producciones son de la forma $A\longrightarrow r$ donde $r$ es un expresión regular de variables y símbolos terminales. Una gramática independiente del contexto generalizada representa una forma compacta de representar una gramática con todas las producciones $A\longrightarrow\alpha$, donde $\alpha$ es una palabra del lenguaje asociado a la expresión regular $r$ y $A\longrightarrow r$ es una producción de la gramática generalizada. Observemos que esta gramática asociada puede tener infinitar producciones, ya que una expresión regular puede representar un lenguaje con infinitar palabras. El concepto de lenguaje generado por una gramática generalizada se define de forma análoga al de las gramáticas independientes del contexto, pero teniendo en cuenta que ahora puede haber infinitas producciones. Demostrar que el lenguaje es independiente del contexto si y solo si se puede generar por una gramática generalizada.
\section*{Ejercicio 17}
Demostrar que los siguientes lenguajes son independientes del contexto:
\begin{enumerate}
\item $L_1=\{u\#v|u^{-1} \text{ es una subcadena de }w, \quad u,w\in\{0,1\}^*\}$
\[S\longrightarrow SX|A\]
\[X\longrightarrow 0|1\]
\[A\longrightarrow 0A0|1A1|\#B\]
\[B\longrightarrow XB\]
\item $L_2=\{u_1\# u_2\# ...\# u_k|k\geq 1, \quad \text{cada } u_i\in\{0,1\}^*, \text{ y para algún }i,j, \quad u_i=u_j^{-1}\}$
\[ S\longrightarrow X\#S|AB\]
\[ X\longrightarrow 0X|1X|\varepsilon \]
\[ A\longrightarrow 0A0|1A1|\varepsilon|\#|\#C\#\]
\[ C\longrightarrow X\#C|X\]
\[ B\longrightarrow \#XB|\varepsilon\]
\end{enumerate}
\section*{Ejercicio 19}
Sea el lenguaje $L=\{u\#v||u,v\in\{0,1\}^*, \enskip u\neq v \}$, demostrar que es independiente del contexto.
\section*{Ejercicio 21}
Dar gramáticas independientes del contexto no ambiguas para los siguientes lenguajes sobre el alfabeto $\{0,1\}$:
\begin{enumerate}
\item  El conjunto de palabras $w$ tal que en todo prefijo de $w$ el número de 0s es mayor o igual que el número de 1s.
\[ S\longrightarrow 0A|\varepsilon\]
\[ A\longrightarrow 1S|0AA|\varepsilon\]
\item El conjunto de palabras $w$ en las que el número d 0s es mayor o igual que el número de 1s.
\[ S\longrightarrow 0A|A0|\varepsilon|A0S0A\]
\[ A\longrightarrow 1|A0A|\varepsilon\]
\end{enumerate}


\end{document}