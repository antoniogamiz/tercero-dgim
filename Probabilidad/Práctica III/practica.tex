
% plantilla obtenida de: https://www.overleaf.com/19886281jjqffwsxshmm#/73112823/

\documentclass[a4paper, 11pt]{article}
\usepackage{comment} % enables the use of multi-line comments (\ifx \fi) 
\usepackage{lipsum} %This package just generates Lorem Ipsum filler text. 
\usepackage{fullpage} % changes the margin

\usepackage[spanish]{babel}
\usepackage[utf8]{inputenc}
\decimalpoint
\usepackage{graphicx}

\usepackage{amsmath}
\usepackage{amsfonts}
% or
\usepackage{amssymb}
\usepackage{tikz}
\usepackage{array}
\newcolumntype{C}{>{$}l<{$}} % math-mode version of "l" column type
\newcommand{\imageins}[4]{\begin{figure}[!ht]		%Take the hardwork from using images. Let this command do the work for you. Insert images by just using this command \imageins{filename}{width as a ratio of total text width of the page}{caption name}{label name for referring in articles}		
    \centering
    \includegraphics[width=#2\textwidth]{#1}
    %\caption{#3}
    %\label{#4}
    \vspace{0.2em}
\end{figure}}

%%%%%%%%%%%%%%%%%%%%%%%%%%%%%%%%%%%%%%%%%%%%%%%%%%%%%%%%%%%%%%%%%%%%%%%%%%%%%%%%%%
\usepackage{listings}
\usepackage{color}
 
\definecolor{codegreen}{rgb}{0,0.6,0}
\definecolor{codegray}{rgb}{0.5,0.5,0.5}
\definecolor{codepurple}{rgb}{0.58,0,0.82}
\definecolor{backcolour}{rgb}{0.95,0.95,0.92}
 
\lstdefinestyle{mystyle}{
    backgroundcolor=\color{backcolour},   
    commentstyle=\color{codegreen},
    keywordstyle=\color{magenta},
    numberstyle=\tiny\color{codegray},
    stringstyle=\color{codepurple},
    basicstyle=\footnotesize,
    breakatwhitespace=false,         
    breaklines=true,                 
    captionpos=b,                    
    keepspaces=true,                 
    numbers=left,                    
    numbersep=5pt,                  
    showspaces=false,                
    showstringspaces=false,
    showtabs=false,                  
    tabsize=2
}
 
\lstset{style=mystyle}

%%%%%%%%%%%%%%%%%%%%%%%%%%%%%%%%%%%%%%%%%%%%%%%%%%%%%%%%%%%%%%%%%%%%%%%%%%%%%%%%%%

\begin{document}
%Header-Make sure you update this information!!!!
\noindent
\large\textbf{Práctica II} \hfill \textbf{Antonio Gámiz Delgado} \\
\normalsize Probabilidad \hfill 19 /11/2018
%\normalsize ECE 100-003 \hfill Teammates: Student1, Student2 \\
%Prof. Oruklu \hfill Lab Date: XX/XX/XX \\
%TA: Adam Sumner \hfill Due Date: XX/XX/XX

\section*{Ejercicio 1}

Hallar la función generatriz de momentos del vector aleatorio discreto $(X,Y)$ con función masa de probabilidad dad en la tabla adjuntada junto a tu DNI/pasaporte, y a partir de ella, calcular la matriz de varianzas-covarianzas.   

\[
\begin{array}{|c|c|c|c|}
\hline
X/Y & -3 & 3 & 5 \\
\hline
0&1/16&1/16&1/8\\
\hline
1&1/2&0&1/4 \\
\hline
\end{array}
\]

Como sabemos, la función generatriz de momentos se calcula como:
\[
M(t_1,t_2) = \sum_{i=0}^1\sum_{j=0}^2e^{t_1x_i+t_2y_j}P[X=x_i, Y=y_j] = 
\]
\[
= \frac{e^{-3t_2}}{16} + \frac{e^{3t_2}}{16} + \frac{e^{5t_2}}{8} + \frac{e^{t_1-3t_2}}{2} + \frac{e^{t_1+5t_2}}{4}
\]

Calculemos ahora la matriz de covarianzas:

\[
E[x] = m_{10} = \frac{\partial M(t_1, t_2)}{\partial t_1}\Big|_{t_1=t_2=0}=\frac{e^{t_1-3t_2}}{2} + \frac{e^{t_1+5t_2}}{4} \Big|_{t_1=t_2=0} = \frac{3}{4}
\]

\[
E[x^2] = m_{20} = \frac{\partial^2 M(t_1, t_2)}{\partial^2 t_1}\Big|_{t_1=t_2=0}=\frac{e^{t_1-3t_2}}{2} + \frac{e^{t_1+5t_2}}{4} \Big|_{t_1=t_2=0} = \frac{3}{4}
\]

Luego la varianza de $X$ será: 
\[
Var[x] = E[x^2]-E[x]^2=\frac{3}{4}-\frac{9}{16}=\frac{3}{16}
\]

\[
E[y] = m_{01} = \frac{\partial M(t_1, t_2)}{\partial t_2}\Big|_{t_1=t_2=0}=\frac{-3e^{-3t_2}}{16} + \frac{3e^{3t_2}}{16} + \frac{5e^{5t_2}}{8} + \frac{-3e^{t_1-3t_2}}{2} + \frac{5e^{t_1+5t_2}}{4} \Big|_{t_1=t_2=0} = \frac{3}{8}
\]

\[
E[y^2] = m_{02} = \frac{\partial^2 M(t_1, t_2)}{\partial^2 t_2}\Big|_{t_1=t_2=0}=\frac{9e^{-3t_2}}{16} + \frac{9e^{3t_2}}{16} + \frac{25e^{5t_2}}{8} + \frac{9e^{t_1-3t_2}}{2} + \frac{25e^{t_1+5t_2}}{4} \Big|_{t_1=t_2=0} = 15
\]

Luego la varianza de $X$ será: 
\[
Var[Y] = E[Y^2]-E[Y]^2=15-\frac{9}{64}=\frac{951}{64}
\]

\[
E[x,y]=m_{11}=\frac{\partial M(t_1, t_2)}{\partial t_1\partial t_2}\Big|_{t_1=t_2=0} = \frac{\partial M(t_1,t_2)}{\partial t_2}\left(\frac{e^{t_1-3t_2}}{2} + \frac{e^{t_1+5t_2}}{4}\right) \Big|_{t_1=t_2=0} = 
\]

\[
= \frac{-3e^{t_1-3t_2}}{2} + \frac{5e^{t_1+5t_2}}{4}  \Big|_{t_1=t_2=0} = -\frac{1}{4}
\]

Luego la covarianza de $X$ e $Y$ será:
\[
Cov[x,y]=E[x,y]-E[x]E[y]=\frac{1}{4}-\frac{3}{4}\frac{3}{8}=-\frac{1}{32}
\]

Ya tenemos la matriz de covarianzas:

\[
Cov_{xy} = \left( \begin{array}{cc}
\displaystyle\frac{3}{16} & -\displaystyle\frac{1}{32} \\
-\displaystyle\frac{1}{32} & \displaystyle\frac{951}{64}
\end{array} \right)
\]

\section*{Ejercicio 2}
Calcular la recta y la curva de regresión de $Y$ sobre $X$ del vector aleatorio continuo ($X$,$Y$) con función de densidad:
\[
f(x,y) = \frac{1}{50}, \quad -2<x<8, \enskip 3-x<y<5
\]
Además calcular el coeficiente de determinación lineal y la razón de correlación.\\[0.5cm]
Empezamos por calcular las curvas de regresión, primero la de $X$ sobre $Y$:
\[
f_2(y)=\int_{-2}^8\frac{1}{50}dx=\frac{1}{5} \quad -5<y<5 \Longrightarrow 
\]\[
\Longrightarrow f(x|y=y_0)=\frac{f(x,y_0)}{f_2(y_0)}= \frac{\frac{1}{50}}{\frac{1}{5}}=\frac{1}{10} \Longrightarrow
\]
\[
\Longrightarrow E[x|y_0]=\int_{-\infty}^{+\infty}xf(x|y=y_0)dx=\int_{-2}^{8}\frac{x}{10}dx=1 \quad -5<y<5.
\]

Ahora la de $Y$ sobre $X$:
\[
f_1(x)=\int_{3-x}^5\frac{1}{50}dy=\frac{x+2}{50} \quad -2<x<8 \Longrightarrow 
\]\[
\Longrightarrow f(y|x=x_0)=\frac{f(x_0,y)}{f_1(x_0)}= \frac{\frac{1}{50}}{\frac{x+2}{50}}=\frac{1}{x+2} \Longrightarrow
\]
\[
\Longrightarrow E[y|x_0]=\int_{-\infty}^{+\infty}yf(y|x=x_0)dx=\int_{3-x}^{5}\frac{y}{x+2}dy=\frac{25-(3-x)^2}{2(x+2)} \quad -2<x<8.
\]

Por lo que nuestras curvas de regresión son:
\[
y=\varphi(x)=E[Y/X]=\frac{25-(3-x)^2}{2(x+2)}  \quad x=\varphi(y)=E[X/Y]=1
\]

Para las rectas de regresión necesitamos:
\[
E[x]=\int_{-\infty}^{+\infty}\int_{-\infty}^{+\infty}xf(x,y)dydx=\int_{-2}^{8}\int_{3-x}^{5}\frac{x}{50} dydx=\frac{14}{3}
\]
\[
E[y]=\int_{-\infty}^{+\infty}\int_{-\infty}^{+\infty}yf(x,y)dydx=\int_{-2}^{8}\int_{3-x}^{5}\frac{y}{50} dydx=\frac{5}{3}
\]
\[
E[x^2]=\int_{-\infty}^{+\infty}\int_{-\infty}^{+\infty}x^2f(x,y)dydx=\int_{-2}^{8}\int_{3-x}^{5}\frac{x^2}{50} dydx=\frac{82}{3}
\]
\[
E[y^2]=\int_{-\infty}^{+\infty}\int_{-\infty}^{+\infty}y^2f(x,y)dydx=\int_{-2}^{8}\int_{3-x}^{5}\frac{y^2}{50} dydx=\frac{25}{3}
\]
\[
E[xy]=\int_{-\infty}^{+\infty}\int_{-\infty}^{+\infty}xyf(x,y)dydx=\int_{-2}^{8}\int_{3-x}^{5}\frac{xy}{50} dydx=5
\]
\[
Var[X]=E[x^2]-E[x]^2=\frac{82}{3}-\left(\frac{14}{3}\right)^2=\frac{50}{9}
\]
\[
Var[Y]=E[y^2]-E[y]^2=\frac{25}{3}-\left(\frac{5}{3}\right)^2=\frac{50}{9}
\]
\[
Cov(X,Y)=E[xy]-E[x]E[y]=5-\frac{14}{3}\frac{5}{3}=-\frac{25}{9}
\]

Recta de regresión de $Y$ sobre $X$:
\[
a=\frac{Cov(X,Y)}{Var[y]}=\frac{\frac{-25}{9}}{\frac{50}{9}}=-\frac{1}{2} \quad b=E[y]-aE[x]=\frac{5}{3}+\frac{1}{2}\frac{14}{3}=4 \Longrightarrow y=-\frac{x}{2}+4
\]
Recta de regresión de $X$ sobre $Y$:
\[
a'=\frac{Cov(X,Y)}{Var[x]}=\frac{\frac{-25}{9}}{\frac{50}{9}}=-\frac{1}{2} \quad b'=E[x]-aE[y]=\frac{14}{3}+\frac{1}{2}\frac{5}{3}=\frac{11}{2} \Longrightarrow x=-\frac{y}{2}+\frac{11}{2}
\]

Coeficiente de determinación lineal:
\[
\rho=\frac{Cov(x,y)}{\sqrt{Var[x]Var[y]}}=\frac{\frac{-25}{9}}{\frac{50}{9}\frac{50}{9}}=-\frac{9}{100}
\]

\end{document}
