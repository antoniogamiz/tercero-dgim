\documentclass[a4paper, twoside, 11pt]{report}

\usepackage[spanish]{babel}
\usepackage[utf8]{inputenc}
\usepackage[TS1,T1]{fontenc}
\usepackage{fourier, heuristica}
\usepackage{array, booktabs}
\usepackage{graphicx}
\usepackage[x11names,table]{xcolor}
\usepackage{color}
\usepackage{caption}
\DeclareCaptionFont{blue}{\color{LightSteelBlue3}}

\newcommand{\foo}{\color{LightSteelBlue3}\makebox[0pt]{\textbullet}\hskip-0.5pt\vrule width 1pt\hspace{\labelsep}}

\newcommand{\dbname}[1]{\textbf{#1}: }

\begin{document}

\begin{table}
\renewcommand\arraystretch{2}\arrayrulecolor{LightSteelBlue3}
\captionsetup{singlelinecheck=false, font=blue, labelfont=sc, labelsep=quad}
\caption{Lenguajes de Bases de Datos}\vskip -1.5ex
\begin{tabular}{@{\,}r <{\hskip 2pt} !{\foo} >{\raggedright\arraybackslash}p{12cm}}
\toprule
\addlinespace[1ex]
1971 & \dbname{ALPHA} Primer lenguaje de base de datos, propuesto por Edgar F. Codd. \\
1976 & \dbname{QUEL}  Primera aparaición de los lenguajes de bases de datos. \\
1986 & \dbname{\color{red} SQL-86} 	Primera publicaci\'on hecha por ANSI. Confirmada por ISO en 1987. \\
1989 & \dbname{SQL-89}  Revisi\'on menor. \\
1992 & \dbname{SQL-92} Revisi\'on mayor. \\
199x & \dbname{OQL} Lenguaje desarrollado a partir de SQL. \\
1997 & \dbname{MDX} Multidimensional Expressions: lenguaje usado para procesamiento analítico online. \\
1998 & \dbname{NoSQL} Not Only SQL. \\
1999 & \dbname{SQL:1999} Se agregaron expresiones regulares, consultas recursivas (para relaciones jer\'arquicas), triggers y algunas caracter\'isticas orientadas a objetos. \\
2003 & \dbname{SQL:2003} Introduce algunas caracter\'isticas de XML, cambios en las funciones, estandarizaci\'on dle objeto sequence y de las columnas autonum\'ericas. \\
2005 & \dbname{SQL:2005} Define las maneras en las cuales SQL se puede utilizar conjuntamente con XML. Define maneras de importar y guardar datos XML en una base de datos SQL, manipul\'andolos dentro de la base de datos y publicando el XML y los datos SQL convencionales en forma XML. \\
2007 & \dbname{.QL} .Query Language, usado para el manejo de sistemas de bases de datos relacionales. \\
2008 & \dbname{SQL:2008} Permite el uso de la cl\'ausula ORDER BY fuera de las definiciones de los cursores. Incluye los disparadores del tipo INSTEAD OF. Añade la sentencia TRUNCATE.\\
2011 & \dbname{SQL:2011} Datos temporales (PERIOD FOR). Mejoras en las funciones de ventana y de la cl\'ausula FETCH. \\
2016 & \dbname{SQL:2016} Permite b\'usqueda de patrones, funciones de tabla polim\'orficas y compatibilidad con los ficheros JSON. \\

\end{tabular}
\end{table}

\begin{center}
\large \textbf{LENGUAJES DE BASES DE DATOS \\ }
\scriptsize Autores: Samuel Medina Gutiérrez, Francisco Vázquez Escobar, Antonio Gámiz Delgado, Laura Sánchez Parra.
\end{center}

\section*{ \large ALPHA}
Fue el lenguaje propuesto por Edgar F. Codd, inventor de las bases de datos relacionales. Este lenguaje influenció el diseño de \textbf{QUEL}. 
\section*{\large QUEL}
Es un lenguaje relacional a través de consultas (queries), basado en el cálculo relacional de tuplas, con algunas similitudes a \textbf{SQL}. Fue creado como por \textit{Ingres DBMS} en la Universidad de California, basado en la previa pero no implementada sugerencia de Codd: \textit{Data Sub-Language ALPHA}. Esta especificación fue posteriormente abandonada y los que la usaban se pasaron a SQL. 
\section*{\large SQL}
SQL (Structured Query Language), creado por \textbf{IBM} en la década de los 70', siendo una combinación de álgebra relacional y cálculo relacional. Siendo publicada su primera estandarización SQL-86, en el año 1986 por ANSI. Este lenguaje consta de 3 partes: \textbf{DML} (Data Manipulation Language), \textbf{DDL} (Data Definition Language) y \textbf{DCL} (Data Control Language).
\section*{\large OQL}
Object Query Language (Object Query Language) es un lenguaje de consultas estándar para bases de datos dirigidas a objetos, modelada después de SQL. Fue desarrollado por el \textbf{Object Data Management Group} (ODMG). Debido a su complejidad, nadie lo ha implementado completamente.
\section*{\large NoSQL}
A diferencia del lenguaje SQl (estructurado), el lenguaje NoSQL puede dinámicamente estructurar los datos dinámicamente como más convenga.

\end{document} 